\documentclass[
  % Babel language, also used to load translations
  english,
  % Use A4 paper size, you can change this to eg. letterpaper if you need
  % the letter format. The normal methods to modify the paper size should
  % be picked up by SDAPS automatically.
  a4paper, % setting this might break the example scan unfortunately
  % letterpaper
  %
  % If you need it, you can add a custom barcode at the center
  %globalid=ScientificWorking,
  %
  % And the following adds a per sheet barcode at the bottom left
  %print_questionnaire_id,
  %
  % You can choose between twoside and oneside. twoside is the default, and
  % requires the document to be printed and scanned in duplex mode.
  twoside,
  %
  % With SDAPS 1.1.6 and newer you can choose the mode used when recognizing
  % checkboxes. valid modes are "checkcorrect" (default), "check" and
  % "fill".
  checkmode=checkcorrect,
  %
  % The following options make sense so that we can get a better feel for the
  % final look.
  %pagemark,
%put final/draft to the pdf to typeset them
  stamp,draft,pdf
]{sdaps}
\ProvidesDictionary{translator-sdaps-dictionary}{English}

\providetranslation{infotext}{This questionnaire can be read by a computer program. Please use a pen for filling in your answers.}
\providetranslation{standard-deviation}{Standard-Deviation}
\providetranslation{info-cross}{Check}
\providetranslation{info-correct}{Uncheck to correct}
\providetranslation{answers}{Answers}
\providetranslation{questionnaireid}{Questionnaire-ID:}
\providetranslation{surveyid}{Survey-ID:}
\providetranslation{draft}{draft}
\providetranslation{info-select}{}
\providetranslation{info-mark}{}
\providetranslation{mean}{Mean}

\usepackage[utf8]{inputenc}
% For demonstration purposes
\usepackage{multicol}

%\author{\choiceitemtext{1.2cm}{2}{Speaker:}}
\title{Scientific working presentation evaluation sheet}

\begin{document}
  % Everything you do should be done inside the questionnaire environment.

  % If you don't like the default text at the beginning of each questionnaire
  % you can remove it with the optional [noinfo] parameter for the environment 
  \begin{questionnaire}
    % There is a predefined "info" style to hilight some text.
\begin{multicols}{3}
Presenter
\textbox*{1cm}{}\vfill\columnbreak
Paper
\textbox*{1cm}{}\vfill\columnbreak
Reviewer
\textbox*{1cm}{}\vfill\columnbreak
\end{multicols}
    % Use \addinfo to add metadata (which is printed on the report later on)
    %\addinfo{Date}{10.03.2013}

    % You can structure the document using sections. You should not use
    % subsections yourself, as these are used to typeset question text.
    \section{Invention}
    \begin{choicegroup}{}
      % We have to add the possible choices at the start.
      \groupaddchoice{Great}
      \groupaddchoice{Partial}
      \groupaddchoice{Poor}
      % After that it is possible to add each question.
      \choiceline{Paper was in the scope of the class and lesson (medical engineering)}
      \choiceline{Paper was sufficiently complex for the class}
      \choiceline{Presentation well addresses also the audience without a deep knowledge of the topic}
    \end{choicegroup}

    \section{Arrangement}
    \begin{choicegroup}{}
      % We have to add the possible choices at the start.
      \groupaddchoice{Great}
      \groupaddchoice{Partial}
      \groupaddchoice{Poor}
      % After that it is possible to add each question.
      \choiceline{Is the talk structured in the meaningful and logical way?}
      \choiceline{Was the proper introduction to a topic given?}
      \choiceline{Was the proper conclusion given? Were there summarized clearly important parts of the talk?}
    \end{choicegroup}

   \section{Style}
    \begin{choicegroup}{}
      % We have to add the possible choices at the start.
      \groupaddchoice{Great}
      \groupaddchoice{Partial}
      \groupaddchoice{Poor}
      % After that it is possible to add each question.
      \choiceline{The style of the presentation adds up to the understandability of the topic.}
      \choiceline{Proper incomporation of image/media into the presentation.}
      \choiceline{Meaningful captions of the figures}
      \choiceline{Correct references and citations}
      \choiceline{Use the template of OVGU university.}
    \end{choicegroup}
    
    \section{Memory}
    \begin{choicegroup}{}
      % We have to add the possible choices at the start.
      \groupaddchoice{Great}
      \groupaddchoice{Partial}
      \groupaddchoice{Poor}
      % After that it is possible to add each question.
      \choiceline{Presenter was well prepared for the talk and the flow of the presentation was smooth.}
      \choiceline{Free speach without major support of manuscript.}
      \choiceline{Compliance with a duration of 20 minutes (2 min. tolerance).}
    \end{choicegroup}
    
    \section{Delivery}
    \begin{choicegroup}{}
      % We have to add the possible choices at the start.
      \groupaddchoice{Great}
      \groupaddchoice{Partial}
      \groupaddchoice{Poor}
      % After that it is possible to add each question.
      \choiceline{Good contact and communication with the audience.}
      \choiceline{Posture towards the audience, proper gestures.}
      \choiceline{Speach clarity, articulation, sufficient speach level.}
    \end{choicegroup}
    
    \section{Discussion/other aspects}
    \begin{choicegroup}{}
      % We have to add the possible choices at the start.
      \groupaddchoice{Yes}
      \groupaddchoice{Partly}
      \groupaddchoice{No}
      % After that it is possible to add each question.
      \choiceline{Was the presentation without factical mistakes?}
      \choiceline{Does the presentation provide correct explanation of the main results of the presented paper?}
      \choiceline{Was the speaker well prepared for the discussion?}
    \end{choicegroup}
   
\section{Evaluation}
    \textbox*{1cm}{Total score as a sum of the points, Yes/Great:1 point, Partial/Partly: 0.5 points, Poor/No: 0 points}
    \textbox*{5cm}{Questions}
    \textbox*{5cm}{Feedback}
\vfill
\begin{multicols}{2}
Signature
\textbox*{1cm}{}\vfill\columnbreak
Date
\textbox*{1cm}{}\vfill\columnbreak
\end{multicols}

    % Reset checkbox style again.
    \def\checkboxstyle{box}

  \end{questionnaire}
\end{document}

