\section{Per-review process}
\begin{frame}
\frametitle{Per-review process}
\begin{itemize}
\item Each student have to send his or her presentation for a review to 25.5.2018.
\item Sending video for extra point.
\item LaTeX presentation for extra point.
\item Reviewer have three weaks to write a review.
\item 15.6.2018 have to be all reviews submitted.
\item He/she has to describe presentation, identify weak points, suggest improvements.
\end{itemize}
\end{frame}

\begin{frame}
\frametitle{Evaluation of presentation proposals by reviewers}
Reviewer have to suggest that the presentation is:
\begin{itemize}
\item \textbf{Acceptable as is} for the best proposals without any flaws.
\item \textbf{Acceptable after corrections} Reviewer would like to see some improvements or clarifications and after they will be done, talk is acceptable.
\item \textbf{Require improvements} For the presentation that can not be presented as is and needs substantial improvements.
\item \textbf{Require major improvements} For the presentation that lacks essential points and needs to be almost completely rewritten.
\end{itemize}
Reviewer must clearly state which parts of presentation should be rewritten or are in his/her opinion wrong.
\end{frame}


\begin{frame}
\frametitle{Response to reviewer}
Student is obligated to:
\begin{itemize}
\item Respond to the all point by the reviewer.
\item If he or she agrees with a reviewer, correct his or her presentation.
\item If he or she does not agree, explain reviewer why the change is not appropriate.
\end{itemize}
Reviewers might have several rounds of iteration until they decide to accept presentation as is. All reviews and responses must be also send to the class teacher. Final decision is on the class teacher.
\end{frame}
